\begin{abstract}
It is very common to face productivity issues when developing software in a microservice-based architecture. One of the reasons is the need to run multiple microservices at the same time to iterate, test and validate some business flow. The computing resources needed for such tasks are, more often than not, unavailable on the developer machine.

This context made possible for a partnership between Poli and Nubank to flourish, looking to tackle these problems. The project was co-supervised by one of the most prestigious Nubank's engineers, responsible for conceiving a big share of the enterprise's software architecture.

Using Cloud Computing, container orchestration, Distributed Tracing and file sync concepts, this solution aims to increase the productivity of Nubank Engineers in their day-to-day interactions with microservices.

To come up with it, a deep study of PaaS, microservices and the company's software development \textit{status quo} was done.

The result is a real-time enviroment for development of large scale applications based on microservices: Formicarium.

It was introduced in a squad of the company for testing and gathering of feedback, and it managed to fulfill the requirements. It was well-received among engineers, and further diffusion is not a complicated task. Various opportunities of continuity are possible as well. 

%
% \\[3\baselineskip]
%

\textbf{Keywords} -- Microservice, Docker, Kubernetes, Cloud Computing, Distributed Tracing.
\end{abstract}