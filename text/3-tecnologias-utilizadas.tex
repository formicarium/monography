\chapter{Tecnologias Utilizadas}
	\section{Clojure}
	
	Clojure é uma linguagem de programação dinâmica e funcional. Sendo um dialeto de LISP que roda na JVM, é uma linguagem compilada, mas que mantém todas características dinâmicas a ser utilizadas em Runtime. Possui compatibilidade e interoperabilidade com todo ecossistema Java e de outras linguagens que rodem na JVM. Clojure enfatiza o uso de estruturas de dados imutáveis e de a filosofia de Code is Data, com um sistema poderoso de macros e de estruturas de dados mutáveis Thread Safe quando necessário. \cite{clojurerationale}
	
	Clojure é a principal linguagem de programação utilizada no Nubank, e também é muito familiar para o grupo, além de apresentar características interessantes para o desenvolvimento do projeto. como a grande habilidade para lidar com problemas de concorrência de forma segura e eficiente.
	
	Dentro do desenvolvimento do projeto Clojure foi utilizada como uma das principais linguagens de programação para serviços de Backend. Um dos recursos que torna Clojure especialmente interessante é a possível interação em tempo real com o código através de um REPL (Read Eval Print Loop), algo que foi utilizado em um dos módulos do projeto como será descrito mais a frente. Além disso a familiaridade dos desenvolvedores do Nubank com a linguagem foi um fator decisivo para que o projeto pudesse continuar a ser mantido por mais pessoas no futuro.
	
	\begin{figure}
	    \centering
	    \includegraphics[scale=0.1]{pictures/clojure_logo.png}
	    \caption{Logo Clojure}
	    \legend{Fonte: \url{https://cdn-images-1.medium.com/max/1200/1*eLqeIits5crU3G5b9LMEyg.png}}
	    \label{fig:logo_clojure}
	\end{figure}
	
	\section{Javascript}
    	\subsection{Typescript}
    	\subsection{NodeJS}
    	\subsection{React}
    	\subsection{Electron}
	\section{ZeroMQ}
	\section{Docker}
	\section{Kubernetes}
	