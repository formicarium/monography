\begin{resumo}
	Em uma arquitetura distribuída de microsserviços, é comum enfrentar problemas relacionados à produtividade enquanto o software é desenvolvido. Uma das causas deste problema é a necessidade de se executar diversos microsserviços ao mesmo tempo para iterar, testar e validar algum fluxo de negócio. Os recursos computacionais demandados para tais tarefas em geral não estão disponíveis na máquina do desenvolvedor.
	
	Este contexto permitiu que uma parceria entre a Poli e o Nubank surgisse, buscando mitigar esses problemas. O projeto foi co-orientado por um dos engenheiros mais prestigiados do Nubank, responsável pela concepção de grande parte da arquitetura de software adotada na empresa.
	
%%	O Nubank é uma fintech brasileira que conta com mais de 200 engenheiros, trabalhando diariamente em mais de 200 microsserviços para atender os mais de 5 milhões de clientes ativos em seus diversos produtos. A empresa é referência nacional e mundial em tecnologia, devido a sua arquitetura de software moderna, baseada em microsserviços. Porém, as implicações negativas deste modelo já estão se manifestando na empresa há algum tempo, prejudicando a produtividade das equipes.
	
	Usando conceitos de computação em nuvem, orquestração de contêineres, Distributed Tracing e técnicas para sincronização de sistemas de arquivos, esta solução visa aumentar a produtividade dos engenheiros do Nubank no cotidiano de interação com microsserviços.
	
	Para tanto, foi feito um estudo profundo sobre PaaS, microsserviços, e o \textit{status quo} de desenvolvimento de software no Nubank. 
	
	O resultado foi um ambiente de desenvolvimento em tempo real para desenvolvimento de microsserviços: Formicarium.
	
	O projeto foi implantado em um squad do Nubank para teste e coleta de feedbacks, e cumpriu os requisitos. Foi bem recebido pelos engenheiros, e uma maior difusão não é uma tarefa complicada. Várias oportunidades de continuidade são possíveis.
	
%
% \\[3\baselineskip]
%
\textbf{Palavras-Chave} -- Microsserviço, Docker, Kubernetes, Computação em Nuvem, Distributed Tracing.
\end{resumo}
