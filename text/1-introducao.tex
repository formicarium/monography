\chapter{Introdução}
	\section{Objetivo}
	%% O que o seu projeto resolve?
	O objetivo deste trabalho é criar um ambiente e um ferramental para acelerar a iterações no desenvolvimento de microsserviços em uma arquitetura de larga escala.
	\section{Motivação}
	%% Por que fazer um trabalho?
	A motivação desse projeto vem de alguns fatores. O primeiro é o interesse e a familiaridade do grupo quanto ao desenvolvimento de microsserviços provenientes da experiência adquirida nos estágios na Nubank, e o reconhecimento dos benefícios e dificuldades essa arquitetura. Outro fator importante é a constatação de que, com o crescimento das aplicações em nuvem computacional, a necessidade e urgência de uma arquitetura distribuída aumenta, e a carência de ferramentas para conter suas dificuldades se torna um problema cada vez mais importante.
	
	Com isso, foi possível a identificação de um problema relativamente complexo de Engenharia de Software que ainda não teve a devida atenção no mercado, e uma oportunidade para aplicação e aprofundamento do nosso conhecimento de processo de desenvolvimento de software. Nós queríamos, então, criar uma proposta de solução para um problema real ainda em discussão.
	\section{Justificativa}
	%% Por que fazer esse trabalho? 
	%% 1. O tema é necessário (debugging microservices)
	O crescimento da arquitetura de microsserviços é notório e pode ser percebido de várias formas. Tanto no mercado quanto na academia, a discussão só tem crescido, com cada vez mais empresas grandes decidindo adotar essa tecnologia, tais como Uber e Netflix. É possível notar também um crescente número de livros que tratam sobre o assunto, como \textit{Production-Ready Microservices} da Susan J. Fowler\cite{productionreadyms} e \textit{Microservices Patterns} de Chris Richardson\cite{mspatterns}. Esse crescimento permitiu a escalabilidade dessas empresas, mas também veio com um custo: a crescente complexidade.
	
	É nesse contexto que o Nubank se encontra e faz parceria neste trabalho. O Nubank é uma fintech brasileira que conta com cerca de 200 engenheiros, trabalhando diariamente em mais de 200 microsserviços para atender os mais de 5 milhões de clientes ativos em seus diversos produtos. A empresa é referência nacional e mundial em tecnologia, devido a sua arquitetura de software moderna, baseada em microsserviços. Porém, as implicações negativas deste modelo já estão se manifestando na empresa há algum tempo, prejudicando a produtividade das equipes. Assim, tivemos a oportunidade de firmar essa parceria e trabalharmos em um contexto real, para resolver uma carência crescente na empresa.
	
	No entanto, este trabalho foi feito de tal forma que, com algum esforço de implantação, as ferramentas desenvolvidas possam ser usufruídas por outras empresas que tenham características de arquitetura semelhantes. Um estudo feito na Universidade de Brighton, no Reino Unido, indica que dentre 33 publicações referentes à arquitetura de microsserviços, o desafio mais citado é referente a comunicação e integração dos microsserviços.\cite{systematicmapping}
	%% 4. Há evidência de que outros contextos podem se beneficiar desse projeto.
	%%\cite{fowlermicroservices}
	\section{Organização do Trabalho}
	%% Como ler esse texto?
	A seguir é descrita a organização do restante do documento
	
	\begin{itemize}
      \item Capítulo 2 - Aspectos Conceituais: Neste capítulo, os principais conceitos utilizados neste trabalho estão explicados com o nível de abstração necessário para o entendimento do restante do documento, principalmente os capítulos referentes à implementação. Os conceitos são apresentados de maneira agnóstica às tecnologias de fato escolhidas para o projeto.
      
      \item Capítulo 3 - Tecnologias Utilizadas: Neste capítulo são apresentadas, juntamente com uma descrição de seus elementos e funcionamento, as tecnologias mais relevantes escolhidas para o projeto.
      
      \item Capítulo 4 - Metodologia de trabalho: Neste capítulo, apresenta-se a metodologia para desenvolvimento deste trabalho.
      
      \item Capítulo 5 - Especificação do projeto: Neste capítulo, está a especificação do projeto feita a partir das visões do RM-ODP.
      
      \item Capítulo 6 - Projeto e Implementação: Neste capítulo, estão apresentados todos os módulos que compõe o projeto e uma descrição detalhada do que fazem e das estratégias adotadas para atender os requisitos do projeto.
      
      \item Capítulo 7 - Testes e Avaliação: Neste capítulo, os testes e a avaliação do sistema são expostos. Experiências reais de uso da ferramenta no ambiente no Nubank são apresentadas.
      
      \item Capítulo 8 - Considerações finais: Neste capítulo, estão as considerações finais deste projeto, incluindo a avaliação do cumprimento dos objetivos, sugestões para trabalhos futuros e uma conclusão geral.
    \end{itemize}
	