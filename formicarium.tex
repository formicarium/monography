\documentclass[]{politex}
% ========== Opções ==========
% pnumromarab - Numeração de páginas usando algarismos romanos na parte pré-textual e arábicos na parte textual
% abnttoc - Forçar paginação no sumário conforme ABNT (inclui "p." na frente das páginas)
% normalnum - Numeração contínua de figuras e tabelas 
%	(caso contrário, a numeração é reiniciada a cada capítulo)
% draftprint - Ajusta as margens para impressão de rascunhos
%	(reduz a margem interna)
% twosideprint - Ajusta as margens para impressão frente e verso
% capsec - Forçar letras maiúsculas no título das seções
% espacosimples - Documento usando espaçamento simples
% espacoduplo - Documento usando espaçamento duplo
%	(o padrão é usar espaçamento 1.5)
% times - Tenta usar a fonte Times New Roman para o corpo do texto
% noindentfirst - Não indenta o primeiro parágrafo dos capítulos/seções


% ========== Packages ==========
\usepackage[utf8]{inputenc}
\usepackage{amsmath,amsthm,amsfonts,amssymb}
\usepackage{graphicx,cite,enumerate}


% ========== Language options ==========
\usepackage[brazil]{babel}
%\usepackage[english]{babel}


% ========== ABNT (requer ABNTeX 2) ==========
%	http://www.ctan.org/tex-archive/macros/latex/contrib/abntex2
%\usepackage[num]{abntex2cite}

% Forçar o abntex2 a usar [ ] nas referências ao invés de ( )
%\citebrackets{[}{]}


% ========== Lorem ipsum ==========
\usepackage{blindtext}



% ========== Opções do documento ==========
% Título
\titulo{Ambiente de desenvolvimento real time para arquitetura de microsserviços em larga escala}

% Autor
\autor{Leonardo C. S. Iacovini\\
		Luíz G. Santos\\
		Rafael L. D. R. Santos\\
		Rafael R. Correia}

% Para múltiplos autores (TCC)
%\autor{Nome Sobrenome\\%
%		Nome Sobrenome\\%
%		Nome Sobrenome}

% Orientador / Coorientador
\orientador{Jorge Risco Becerra}
\coorientador{Rafael de França Ferreira}

% Tipo de documento
\tcc{Engenharia de Computação}
%\dissertacao{Engenharia Elétrica}
%\teseDOC{Engenharia Elétrica}
%\teseLD
%\memorialLD

% Departamento e área de concentração
\departamento{PCS}
\areaConcentracao{Engenharia de Software}

% Local
\local{São Paulo}

% Ano
\data{2018}




\begin{document}
% ========== Capa e folhas de rosto ==========
\capa
\falsafolhaderosto
\folhaderosto


% ========== Folha de assinaturas (opcional) ==========
%\begin{folhadeaprovacao}
%	\assinatura{Prof.\ X}
%	\assinatura{Prof.\ Y}
%	\assinatura{Prof.\ Z}
%\end{folhadeaprovacao}


% ========== Ficha catalográfica ==========
% Fazer solicitação no site:
%	http://www.poli.usp.br/en/bibliotecas/servicos/catalogacao-na-publicacao.html


% ========== Dedicatória (opcional) ==========
%\dedicatoria{Dedicatória}


% ========== Agradecimentos ==========
%\begin{agradecimentos}

%Thanks...

%\end{agradecimentos}


% ========== Epígrafe (opcional) ==========
%\epigrafe{%
%	\emph{``Epígrafe''}
%	\begin{flushright}
%		-{}- Autor
%	\end{flushright}
%}


% ========== Resumo ==========
\begin{resumo}
Resumo...
%
\\[3\baselineskip]
%
\textbf{Palavras-Chave} -- Palavra, Palavra, Palavra, Palavra, Palavra.
\end{resumo}


% ========== Abstract ==========
\begin{abstract}
Abstract...
%
\\[3\baselineskip]
%
\textbf{Keywords} -- Word, Word, Word, Word, Word.
\end{abstract}


% ========== Listas (opcional) ==========
%\listadefiguras
%\listadetabelas

% ========== Listas definidas pelo usuário (opcional) ==========
%\begin{pretextualsection}{Lista de símbolos}

%Lista de símbolos...

%\end{pretextualsection}

% ========== Sumário ==========
\sumario



% ========== Elementos textuais ==========

%\part{Introdução}
	
\chapter{Introdução}
	\blindtext
	\section{Objetivo}
		\blindtext
	\section{Motivação}
		\blindtext
	\section{Justificativa}
		\blindtext
	\section{Organização do Trabalho}
		\blindtext

\chapter{Aspectos Conceituais}
	\blindtext

\chapter{Tecnologias Utilizadas}
	\blindtext

\chapter{Metodologia do Trabalho}
	\blindtext

\chapter{Especificação de Requisitos do Sistema}
	\blindtext

\chapter{Projeto e Implementação}
	\blindtext

\chapter{Testes e Avaliação}
	\blindtext

\chapter{Considerações Finais}
	\blindtext
	\section{Conclusões do Projeto de Formatura}
		\blindtext
	\section{Contribuições}
		\blindtext
	\section{Perspectivas de Continuidade}
		\blindtext

%\capepigrafe[0.5\textwidth]{``Frase espirituosa de um autor famoso''}{Autor famoso}

%\begin{citacaoLonga}
%	\blindtext
%\end{citacaoLonga}

\blindtext



%\blinddocument


% ========== Referências ==========
% --- IEEE ---
%	http://www.ctan.org/tex-archive/macros/latex/contrib/IEEEtran
%\bibliographystyle{IEEEbib}

% --- ABNT (requer ABNTeX 2) ---
%	http://www.ctan.org/tex-archive/macros/latex/contrib/abntex2
%\bibliographystyle{abntex2-num}

%\bibliography{}


% ========== Apêndices (opcional) ==========
%\apendice
%\chapter{}
%\chapter{Beta}


% ========== Anexos (opcional) ==========
%\anexo
%\chapter{Alpha}
%\chapter{}



\end{document}
